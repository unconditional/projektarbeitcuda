



\subsection{Multiplikation von Matrize mit Vektor in CUDA Implimentierung}
Multiplikation der Matrize mal Vektor trefft sehr haufig in wissenschaftliche Rechnen auf. Die Operation kann in unterschiedliche Vektor-Multipliktionen zerlegen.

\subsubsection{Fullmatrizemultiplikation}

Die genaue Implimentierung der Vektor-Multiplikation in CUDA wird folgend (Fig ) 
gezeigt. Vektor A wird mit Vektor B multipliziert. 
Man bearbeitet jede Element-Multiplikation in jeweiligem Thread und erzeugt Produktvektor Cs.  
Alle Elements von Cs wudern zur ein Skalarwert summiert. 
Weil CUDA-Blocksize beschrankt ist, 
 NVIDIA CUDA Pro
%gramingGuide_2.3,A.1.1 ), 
maximum 512 Threads per Block, muss man für große Vektoren in meher Blocks oder ein Block iterativ verwenden.

\begin{firuge}[htbp]
\centering
\includegraphics[height=3cm,width=3cm]{.//pic//Vektor}
\caption{Vektor Multiplikation. A : first vector; B:second vector ; Cs: product vector.}
\lable{fig:vektor}
\end{figure}

Bei der Multipliktionen der Matrizen mit Vektoren ist ganz änhlich,dass jede zerlegende Vektor-Multipliktion in ein Block bearbeitet wird.

%\begin{firuge}[htbp]
%\centering
%\includegraphics{.//pic//MatrixVektor}
%\caption{Matriz mal Vektor. A:Matrize; b: Vektor; c: produkt Vektor}
%\lable{fig:vektor}
%\end{figure}


%\subsubsection{Sparsematrizenmultiplikation}

\subsubsection{Sparse-Matrizen}
In vielen Fall bearbeitet man Sparse-Matritzen bei numerische Rechenen. Wie seht eine Sparse-Matrze aus? Wie wird die gespeischert? Fig.1. zeigt auf, wie alle Elements in einer Sparse-Matrize verteilt erden. Für numerische Speischer wird durch eine einfache Idee nur Nonzero-Elements und zugehörige Stelleinfomationen wie in Fig. gespeichert. Der Vektor pr enthalt alle Nonzero-Elements, Vektor ir zugehörige Zeileninformationen, Vektor jc Spaltinformationen. Eine weiter kompakte Struktur ist Fig.3. "compressed column structure", indem die Werte von jc[i] und jc[i+1]-1 die Index von zur Spalte i gehörtet Nonzero-Elements und Zeile aufweisen.

%\begin{figure}[htbp]
%\centering
%\begin{minipage}[t]{0.3\textwidth}
%	\centering
%	\includegraphics{.//pic//orignal_sparse}
%	\caption{Originale-Matrize}
%\end{minipage}
%\begin{minipage}[t]{0.3\textwidth}
%	\centering
%	\includegraphics{.//pic//numerische_sparse1}
%	\caption{Sparsame Struktur}
%\end{minipage}
%\begin{minipage}[t]{0.3\textwidth}
%	\centering
%	\includegraphics{.//pic//numerische_sparse2}
%	\caption{"compressed column structure"}
%\end{minipage}	
%\end{figure}


\subsubsection{SparseMatrixMultiplikation}
%In "compressed column structure " wird Sparsematrizen in Spaltfolg gespeichert.  In unser Implementierung verwendt man in Zeilfolg gespeischerte SparseMatrizen. Folgend Fig zeigt die genau Verfahren dieser Operation.  Wie vorliegend beschreibung besteht Sparsermatrix aus 3 Vektoren Pr, Ir, Jc. Aus Jc findet man zu jede Zeile gehörte Nonzero-Elements und Spalteninfo, die auch  entspreschende Elements aus Vektor b zeigen. Weiter Verfahren ist die Multiplikation der ausgewahlten Nonzero-Elementen  und Elements aus Vektor b, die analog zu Vektormultiplikation.

%\begin{firuge}[htbp]
%\centering
%\includegraphics{.//pic//sparseMul}
%\caption{Sparsematrizenmultiplikation. A:Sparsematrize, Jc:Vektor der Zeileinfo, Pr: Vecktor der Nonzeroelements, Ir: Vektor der Spalteninfo ; b: Vektor; c: produkt Vektor}
%\lable{fig:sparsmultiplikation}
%\end{figure}

\subsection{Perfrmance Optimierung}
Eine weitere Überlegung ist Perfrmance-Optimierung der vorgestellt Operationen in CUDA-Implimentierung. .
Folgend sind audground der Implimentierung der jede Operationen.

%1. Dreieckförmige Summation (Sumierung in Parallel)
%2. Minimierung leer laufende Thread.(32 Thread ein Wrap )
%3. Share Memory (geringer latency als globale Memory)

\subsubsection{Dreieckförmige Summation}
Aus der Beschreibungen der Operationen Multipliktionen der Matrize mal Vektor und Sparsematrize  mal Vektor beruhen obige Operationen auf Vektormultiplikationen, die anschließlichen ein Summierungverfahren in jedem Block enthalten. Blocksummation in einzigen Thread ist nicht effizent. Die einführende Algorithmus: Dreieckförmige Summation lautet wie Fig [] . In erstschritte werden 2n und 2n+1 Elements des Produktvektors Cs in jeweilig Threads summiert. In zweiter schritte werden 4n und 4n+2 Elements summiert. Bis BlockSize/2 Schritte erhalt man endlich Ergebnisse.

%\begin{firuge}[htbp]
%\centering
%\includegraphics{.//pic//dreieck}
%\caption{Dreieckförmige Summation. Von Oben nach Unter zeigt}
%\lable{fig:Dreieckförmige}
%\end{figure}

\subsubsection{Minimierung leer laufende Thread}
Im Cuda, bearbeitet jede Multiprzessors gleichzeitig eine Warp von 32 Threads[cuda programing guide], die alle zur selbe Block gehören. d.h. Für sehr dümm bestzt Matrix, wie in Sparse-Matrix-Multiplikation, sind viele Threads in Leerlauf. Um die Anzahl der in Leer laufende Thread zu minimieren, werden mehre Zeile in eine Block bearbeitet. Dazu verwndet man 2 Dimensionenblocksize. Die Dimension X wird hier für Elementemultiplikationen innerhalb jeweiliger Vektormultipliktionen definiert. Die Dimension Y besorget Unterschiedlich Vektormultiplikationen innerhald eines Blockes. Ein optimierte Beispiel ist Sparsermatrizemultiplikation, deren Implimentierung ähnlich obiger Darstellung ist. Wie bestimmt man die genaue Größe für beide Dimensionen?  Man kann für speziale Anwendungen durch Folgende Vesuche die beide Dimensiongröße auswhalen. Fig[] zeigt die Laufzeit der Multipliktion Ein-Diagonalematrize mal Vektor mit unterschiedliche Blocksize. Es ist offenbar,dass für 1-Diagonalmatize das optimale Blocksize 16x1 oder 32x1 beträgt. 1-Diagonalmatize ist nicht einzige dünbesetzte Matrize in unser Anwendung. Dickere Sparsermatrizen entstehen auch haufig. Fig[] ist die Messung für 32-Diagonalmatize. Man find 16x16 eine schlauere Auswahl.

%\begin{figure}[htbp]
%\centering
%\begin{minipage}[t]{0.3\textwidth}
%	\centering
%	\includegraphics{.//pic//einDiagonal}
%	\caption{Multipliktion Ein-Diagonalematrize mal Vektor. BlockY: Anzahl der Y-Dimension von Block; BlockX: Anzahl der X-Dimension von Block}
%\end{minipage}
%\begin{minipage}[t]{0.3\textwidth}
%	\centering
%	\includegraphics{.//pic//moreDiagonal}
%	\caption{Multipliktion 32-Diagonalematrize mal Vektor. BlockY: Anzahl der Y-Dimension von Block; BlockX: Anzahl der X-Dimension von Block.}
%\end{minipage}
%5\end{figure}

Wie effezient ist denn optimierte CUDA-Implimentierung? 
Weiter Versuch der Vergleichung von matlab,CPU-und GPU-Implimentierung wird in Fig[] ausgewiesen. Für 128-Diagonalmatize kann CUDA-Implimentierung gegen CPU zu Faktor 9 erreichen.
%\begin{firuge}[htbp]
%\centering
%\includegraphics{.//pic//compareSparse}
%\caption{Vergleichung der Sparseermatrize-Multiplikation von matlab,CPU-und GPU-Implimentierung.}
%\lable{fig:compareSparse}
%\end{figure}

\subsubsection{Share Memory}
Im Grafikater ,sind Zugriff der Globalespeicher  langsamer als änder Speicher.  Wie Beispiele in [CUDA Programing Guide] gezeigt, kann man meher mal verwendete Daten zunächst in share Memory schreiben, dann für die entsprechenden Operationen benutzen. In der Multipliktion der Fullmatrize mal Vektor wird jede Vektorelemnt mehr mal gebraut. Nach Untersuchungen wahlen wir 1-Dimensionblock,die 64 beträgt und jede Vektorelement 8 mal gerbraucht in einem Block, d.h. in jedem Block 8 zerlegene Vektormuliplikation bearbeitet werden. Aus den Ergebnise von Fig[](Vergleich von optimierte Fullmatrix-Multiplikatin mit C-Implimentierung und alte GPU-Implimentirung für MxN Fullmatrizen) 

%\begin{firuge}[htbp]
%\centering
%\includegraphics{.//pic//sharememory}
%\caption{Vergleich von optimierte Fullmatrix-Multiplikatin mit C-Implimentierung und alte GPU-Implimentirung für MxN Fullmatrizen}
%\lable{fig:sharememory}
%\end{figure}

Die optimierte GPU-Implimentierung ist immer schnelle als die CPU-Implimentierung  und die Alte. Für Matrize 5000x500 kann die CUDA-Program 30 mal schneller als CPU.