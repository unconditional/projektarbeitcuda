%sparse_matrize

Sparse Matrize\\
Sparsematrizen, oder dünnbesetzte Matrizen,  bezeichnet man als eine Matrizen, bei der so viele Einträge aus Nullen bestehen. Im Fig 1 wird ein einfaches Beispiel gezeigt. 
Da Sparsematrizen mit Vollmatrizen genau umgekehrt ist, hat man dafür auch eine andere Speicherweise. Unter dem Zusammenhang zwischen Fig 1, Fig 2 und Fig 3 versteht man, dass bei der Sparsematrizen wird nun nur die Nonzero-Elementeund und die zugehörigen Stelleinformationen(Zeilen und Spalte) gespeichert. Vektor \textquotedblleft pr \textquotedblright enthält alle Nonzero-Elemente. Die Vektoren \textquotedblleft ir\textquotedblright und \textquotedblleft jc\textquotedblright enthalten die Zeileninformation und die Spalteinformation. Im Fig 3 bezeichnet man, wie die Informationen gepackt werden. Die Werte von \textquotedblleft jc[i]\textquotedblright und \textquotedblleft jc[i+1]-1\textquotedblright zeigen den Indexe, deren die zur Spalte \textquotedblleft i\textquotedblright  gehörteten Nonzero-Elemente und Zeileinformation aufweisen. 

%\begin{figure}[htbp]
%	\includegraphics[width=1in]{.//pic//orignal_sparse}
%	\caption{Original-Matrize}
%	\includegraphics[width=1in]{.//pic//numerische_sparse1}
%	\caption{Sparsame Struktur}
%	\includegraphics[width=1in]{.//pic//numerische_sparse2}
%	\caption{compressed column structure"}
%\end{figure}

\begin{figure}[b]
\setlength{\unitlength}{1cm}
	\begin{minipage}[t]{5.5cm}
	\begin{picture}(5.5,2.5)
	\includegraphics[width=1in]{.//pic//orignal_sparse}
	\end{picture}\par
	\caption{Original-Matrize}
	\end{minipage}\hfill
	\begin{minipage}[t]{5.5cm}
	\begin{picture}(5.5,2.5)
	\includegraphics[width=1in]{.//pic//numerische_sparse2}
	\end{picture}\par
	%\caption{numerische_sparse2}
	\end{minipage}\hfill
\end{figure}