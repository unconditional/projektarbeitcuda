\documentclass[twocolumn]{IEEEtran}
\usepackage{epsf}
\usepackage{./fit}
\usepackage{./forms}

\def\BibTeX{{\rm B\kern-.05em{\sc i\kern-.025em b}
\kern-.08em T\kern-.1667em\lower.7ex\hbox{E}\kern-.125emX}}

\setcounter{page}{1}
\setlength\arraycolsep{1pt}

\newcommand{\fMm}{{\bf M}_{\rm m}}
\newcommand{\fMl}{{\bf M}_{\lambda}}

\begin{document}

\title{Titel der Seminarausarbeitung
%
\thanks{\hrule}%
\thanks{Manuscript received October 25, 1999}%
\thanks{N. Author1 (Matnr.), E. Author2 (Matnr.) are
at the Technische Universit\"at Darmstadt, FB 18 Elektrotechnik
und Informationstechnik, Fachgebiet Theorie Elektromagnetischer Felder,
Schlo\ss{}gartenstr.~8, D-64289~Darmstadt, Germany}%
\thanks{}
\thanks{Email: author1/author2@temf.\-tu\--darm\-stadt.\-de}%
}

\author{M. Author1 and E. Author2}

\markboth{Ausarbeitung zum Seminar {\it Rechnergest\"utzte Methode der Feldberechnung 1} }%
{Thema der Ausarbeitung}

\maketitle

\begin{abstract}
Zusammenfassender Kurztext, nicht mehr als 150 Worte.
\end{abstract}

\begin{keywords}
Stichw\"orter
\end{keywords}

\section{Introduction}
Temperature dependent material properties  and Joule heat losses in
dissipative media imply coupled thermal and electromagnetic processes.
Microwave heating, the self heating of semiconductor devices and the
induction-heat treatment of metals are a few examples of applications,
where such a coupling occurs. For the design and optimization of
these applications, an accurate, simultaneous solution of Maxwell's
equation and of the heat conduction equation is required.

Several {\bf FEM} algorithms for the solution of the coupled thermal
and electromagnetic equations have been suggested (cf.~\cite{Hameyer}).
However, it remains difficult to achieve satisfactory convergence
conditions and consistent numerical results \cite{Kost}. This is
mainly due to the different nature of the two sets of equations,
implying very distinct time scales $t_{\srr EM}$ and $t_{\srr T}$,
characteristic for the electromagnetic and thermal precesses,
respectively.

The Finite Integration ({\bf FI}) Method has been successfully
employed for the solution of Maxwell's equations \cite{Weiland96e:01}.
The resulting Maxwell-Grid-Equations ({\bf MGE}) allow for strict
conservation of charge, momentum and energy, and thus guarantee for the
long-time-stability of the numerical solution. This feature is especially
attractive for the simulation of the slowly changing thermal fields,
which are typically characterized by long transient states. The
{\bf FI} Method  has also been used in the calculation of stationary
thermal and electromagnetic field distributions
\cite{RienenPinderWeiland96e:01}.

The present work extends the previous study \cite{RienenPinderWeiland96e:01}
to the calculation of transient fields. The time dependent heat conduction
equation for isotropic media reads
\beq
\varrho(\bvec{r},T) c(\bvec{r},T)
\dif{T(\bvec{r},t)}{t} = - \di\bvec{J}_w(\bvec{r},t) +
Q_w(\bvec{r},t) \; ,
\label{eqn:1}
\enq
where $\varrho$ and $c$ denote the generally temperature dependent
density and volumetric heat capacity of the medium. The thermal
current density $\bvec{J}_w$ is given by the Fourier law,
\beq
\bvec{J}_w(\bvec{r},t) = - \lambda\grad T(\bvec{r},t)
\label{eqn:2}
\enq
with $\lambda=\lambda(\bvec{r},T)$ the thermal conductivity. Except
for the Joule heat excitation term $Q_w$ appearing in the thermal
equation, also the radiant and convective boundary conditions are
accommodated in the model. The algorithm for the integration in
the time domain makes explicitly use of the different time scales
$t_{\srr EM}$ and $t_{\srr T}$. Thus, unnecessary iterations in the
electromagnetic computation are excluded and the solution speed and
convergence rate are improved. Currently, only temperature dependent
material properties are supported, whereas nonlinear properties of,
e.g., ferromagnetic materials are not yet implemented.

\section{The {\bf FI} Method}
The {\bf FI} discretization scheme is based on a dual grid-doublet
$\{G,\Gt\}$, which decomposes the computation domain into two sets of
dual cells \cite{Weiland96e:01}. Integral quantities $\q$, $\ve$ and
$\fb$ are defined on the grid $G$, corresponding to the total charge
in the cell volumes, to the electric voltage along the cell edges
and to the magnetic induction flux on the cell facets, respectively.
Analogously, $\fj$, $\fd$ and $\vh$ are the vectors of charge current,
electric displacement flux and magnetic voltage defined on the
facets and edges of the dual grid $\Gt$. Fig.~\ref{fig:1}
illustrates the allocation of fluxes and voltages in the
case of rectangular dual grids $G$ and $\Gt$.
\befig[htb]
\centering
\mbox{\epsfxsize=86mm\epsfbox{e-fit.eps}} \\[-5pt]
\caption{{\bf (a)} Two cells of the rectangular dual grids $G$ and
$\Gt$ with given indices $(i,j,k)$ and the allocation of charge $\q$
are shown. {\bf (b)} Allocation of electric voltage $\ve$
and of magnetic induction flux $\fb$ on the direct grid $G$.}
\label{fig:1}
\enfig

Using these integral quantities, Maxwell's equations in discrete form,
the so-called Maxwell-Grid-Equations ({\bf MGE}) are obtained:
\beqarr
\left\{
\begin{array}{llllll}
\C  \ve & = & \displaystyle - \ddif{}{t}\fb \; , \qquad &\Ct \vh
      & = &  \displaystyle \ddif{}{t}\fd + \fj \; ,      \\[5pt]
\displaystyle
\S  \fb &=& {\bf 0} \; ,          & \St \fd &=& \q \; .
\end{array}\right.
\label{eqn:3}
\enqarr
The support matrix operators $(\C,\S)$ and $(\Ct,\St)$  defined
on $G$ and $\Gt$ are discrete mappings of the differential ``$\curl$''
and ``$\di$''. It follows from the topology of the dual grid-doublet
\cite{Weiland96e:01}, that the operators $\C$, $\S$, $\Ct$ and
$\St$ fulfill the identities $\S\C = \C\St^{\rm T} = 0$ and
$\St\Ct = \Ct\S^{\rm T} = 0$, which obviously correspond to the
continuum relations $\di\curl = 0$ and $\curl\grad = \nmbr{0}$. Note,
that equations (\ref{eqn:3}) hold true exactly, thus exact conservation
laws in discrete form for charge, momentum and energy may be
derived \cite{Thoma:Weiland}.

The discretization approximation enters the {\bf FI} Method through
the constitutive material equations
\begin{equation}
\fd=\fMeps\ve + \mbox{\rm\bf p}, ~~~\fj=\fMsig\ve
\quad\mbox{\rm and}\quad \fb=\fMmu\vh + \mbox{\rm\bf m}\; ,
\label{eqn:4}
\end{equation}
which close the system of {\bf MGE} (\ref{eqn:3}) and relate
vectors defined on $G$ and $\Gt$. Here, $\fMeps$, $\fMsig$ and
$\fMmu$ are matrix operators taking into account the average effects of
linear polarization, electric conductivity and linear magnetization of
the material medium. Details on the material and geometry averaging
techniques used with the {\bf FI} Method are found in \cite{Weiland96e:01}.
For dual orthogonal grids $G$ and $\Gt$ the global discretization error
of this approximation is of second order accuracy in the discrete solutions
of (\ref{eqn:3}) and (\ref{eqn:4}).

The discretization of the heat conduction equation is carried out
analogously on the same dual grid-doublet $\{G,\Gt\}$ as above.
Integrating equation (\ref{eqn:1}) over the cell volumes of the grid
$G$ and equation (\ref{eqn:2}) over the cell facets of the dual grid
$\Gt$ one obtains the set of discrete equations,
\beq
\left\{
\begin{array}{l}\displaystyle
\fMm\ddif{}{t}\T=-\St \fj_w  + \q_w \\[5pt]
\displaystyle \fj_w = \fMl\St^{\rm T} \T \; .
\end{array}\right.
\label{eqn:5}
\enq
The vector $\fj_w$ of thermal currents is defined on the facets of the
dual grid $\Gt$, $\T$ is the vector of all temperature values allocated
on the nodes of the grid $G$. Similarly, the vector $\q_w$ of Joule heat
losses is allocated on the nodes of $G$ (see Fig.~\ref{fig:2} for the
allocation in the case of rectangular dual grids). The matrixes $\fMm$
and $\fMl$ contain information on the (cell-averaged) heat capacity
and thermal conductivity of the material medium. Combining equations
(\ref{eqn:1}) and (\ref{eqn:2}) yields
\beq
\fMm\ddif{}{t}\T=-\St\fMl\St^{\rm T} \T + \q_w \; ,
\label{eqn:6}
\enq
which is the discrete thermal equation in the {\bf FI} formulation.
Note, that this formulation implies energy conservation at any given
time $t$. Since the thermal current $\fj_w$ has the same value and
direction on the common facet of two neighboring cells, no energy
loss occurs while heat is transferred from one cell to the other.
\befig[htb]
\centering
\mbox{\epsfxsize=86mm\epsfbox{t-fit.eps}} \\[-5pt]
\caption{{\bf (a)} Allocation of temperature $T$, heat source
${\bf q}_w$ and thermal current $\fj_w$ on a rectangular dual
grid-doublet $\{G,\Gt\}$. {\bf (b)} Allocation of thermal
current on $\Gt$.}
\label{fig:2}
\enfig
\section{Integration in the time domain}
The time domain equivalent to the {\bf FI} Method is the well known
{\bf FDTD} scheme of leapfrog integration. Applied to the time dependent
{\bf MGE} (\ref{eqn:3}) this procedure is restricted by the Courant
stability criterion on the time step length:  $\Delta t \le
\Delta t^{\srr max}_{\srr EM}$, where
\beq
\Delta t^{\srr max}_{\srr EM}=\min_G\left\{\sqrt{\varepsilon_i\mu_i}
\left(\frac{1}{\Delta x_i^2}+\frac{1}{\Delta y_i^2}+\frac{1}{\Delta
z_i^2} \right)^{-\frac{1}{2}} \right\} \; .
\label{eqn:7}
\enq

For the time integration of equation (\ref{eqn:6}) an explicit forward
time difference scheme is used. The corresponding update relation is
\beq
\T^{n+1}=( \One -\Delta t \fMm^{-1} \St \fMl\St^{\rm T} )
\T^{n} + \Delta t\fMm^{-1}\q_w^n \; ,
\label{eqn:8}
\enq
with the maximal stable time step given by
\beq
\Delta t^{\srr max}_{\srr T} =
\min_G\left\{\frac{\rho_i c_i}{2\lambda_i} \left(
\frac{1}{\Delta x_i^2}+\frac{1}{\Delta y_i^2}+\frac{1}{\Delta
z_i^2} \right)^{-1} \right\} \; .
\label{eqn:9}
\enq
Equation (\ref{eqn:9}) is slightly modified, if radiant and convective
boundary conditions are considered (see \cite{Pinder} for details).
For homogeneous material media and equidistant grid points the integration
scheme (\ref{eqn:8}) reduces to Richardson's explicit method for linear
diffusion equations.

Since the stability criterion (\ref{eqn:9}) may become restrictive,
an implicit time integration scheme is used, alternatively. The update
relation for the heat equation in this case reads
\beq
\T^{n+1} = (2\cdot\One + \mbox{\boldmath$\Gamma$})^{-1} (2\cdot\One -
\mbox{\boldmath$\Gamma$})\T^{n}+\Delta t\cdot\fMm^{-1}(\q_w^{n+1}+\q_w^n) \; ,
\label{eqn:10}
\enq
with $\mbox{\boldmath$\Gamma$}= \Delta t\cdot\fMm^{-1}\St\fMl\St^{\rm T}$.
If the material medium is homogeneous and the grid points equidistant,
this procedure corresponds to the well known Crank-Nicolson implicit
time integration scheme. For the solution of the implicit recursion in
(\ref{eqn:10}) a Preconditioned Conjugate Gradient ({\bf PCG}) method
is used.

The calculation effort for the coupled electromagnetic and thermal fields
may be significantly reduced, if the time scales $t_{\srr EM}$ and
$t_{\srr T}$ are taken into account. The electromagnetic time
scale $t_{\srr EM}$ describes, e.g., the time interval until the
steady state distribution of the electromagnetic fields is established.
The thermal time scale $t_{\srr T}$ describes the velocity of diffusion
of the temperature field in absence of external heat sources. If the
same dual grid-doublet $\{G,\Gt\}$ is used for both discretizations
in (\ref{eqn:3}) and (\ref{eqn:6}), then the maximal stable time steps
given in (\ref{eqn:7}) and (\ref{eqn:9}) are appropriate estimations
of the time scales $t_{\srr EM}$ and $t_{\srr T}$, respectively.
For many material media, it is generally observed, that
$\Delta t^{\srr max}_{\srr EM} \ll \Delta t^{\srr max}_{\srr T}$
(e.g., for a copper material block discretized with $\Delta x =
\Delta y = \Delta z = \nmbr{1}\cm$ one obtains
$\Delta t^{\srr max}_{\srr EM} = \nmbr{1.93}\cdot \nmbr{10}^{-11}\secs$
and $\Delta t^{\srr max}_{\srr T} = \nmbr{0.15}\secs$). Therefore,
a stationary electromagnetic field distribution is established long
before significant modifications of the temperature distribution
are observed. For such a weak coupling the algorithm shown in
Fig.~\ref{fig:3} is applicable. Here, the electromagnetic
computation is initiated only if the change in the temperature
distribution is significant, i.e., if the electromagnetic material
properties $\varepsilon(\bvec{r},T)$, $\mu(\bvec{r},T)$ change.
\befig[htb]
\centering
\mbox{\epsfxsize=85mm\epsfbox{diagram.eps}} \\[-5pt]
\caption{Transient solution algorithm for the coupled thermal
and electromagnetic equations.}
\label{fig:3}
\enfig
\section{Validation}
The validity of the method is demonstrated in the calculation
of the transient temperature distribution of a lossy dielectric
rod heated by microwaves. The sample is aligned along the axes
of a cylindrical cavity as shown in Fig.~\ref{fig:4}a. Hybrid heating,
due to microwave absorption as well as to thermal radiation from the
sample surface is considered.
\befig[htb]
\centering
\mbox{\epsfxsize=86mm\epsfbox{cavity.eps}} \\[-5pt]
\caption{{\bf (a)} Lossy dielectric rod in the microwave cavity.
{\bf (b)} Magnetic flux density created by the ground mode
${\bf TM}_{010}$ at $T=25\celcius$ (the resonance frequency
is $\omega_{\rm R}=1.09\GHz$).}
\label{fig:4}
\enfig

Jackson et al.\ \cite{Jackson:01,Jackson:02} developed an analytical
method for calculating the transient temperature profile in this
model. In particular, in the case of a single driving mode of the type
${\bf TM}_{010}$ the spatial variation of both, thermal and electromagnetic
fields is on the radial direction alone and analytical results are
easier to obtain. The sample discussed in \cite{Jackson:01,Jackson:02}
was an alumina rod of radius $\nmbr{1.87}\cm$, length $\nmbr{6.63}\cm$
and thermal emissivity $\nmbr{0.31}$. The radius of the cavity was
$\nmbr{4.69}\cm$ and the cavity walls were considered as perfectly
absorbing at a constant temperature of $\nmbr{25}\celcius$. The amplitude
of the driving mode ${\bf TM}_{010}$ corresponded to a constant input
power of $\nmbr{400}\watt$. The temperature dependent values of
$\varepsilon$ and $\lambda$ for nominal alumina materials refer to
the experimental data given in \cite{Fukushima}.

Figures~\ref{fig:5}-\ref{fig:7} show the results of the numerical
simulation with the {\bf FI} Method. Fig.~\ref{fig:5} shows the
transient temperature profile on the rod axes until the stationary
state is established. The relative deviation from the analytical
results (see Fig.~\ref{fig:6}) remains well below $\nmbr{1}\%$.
Fig.~\ref{fig:7} shows the electric field strength inside the cavity
at different times. Also in this calculation the deviation from the
reference solution is negligible. Additional results related to
the microwave heating of the cylindrical sample, including runaway
heating effects, are given in \cite{Jackson:01} and \cite{Pinder}.
\befig[htb]
\centering
\mbox{\epsfxsize=68mm\epsfbox{t-center.eps}} \\[-5pt]
\caption{Temperature profile on the rod axes as a function of time.}
\label{fig:5}
\enfig
\befig[htb]
\centering
\mbox{\epsfxsize=68mm\epsfbox{t-error.eps}} \\[-5pt]
\caption{Relative deviation $\Delta T/T$ from the reference solution.}
\label{fig:6}
\enfig
\befig[htb]
\centering
\mbox{\epsfxsize=68mm\epsfbox{e-rod.eps}} \\[-5pt]
\caption{Electric field strength $|E|$ inside the cavity as a function
of the radial coordinate $r$.}
\label{fig:7}
\enfig
\section{Transient microwave heating of a jelly block}
As a second application of the method, the heating of a jelly
block in a PTFE container inside a rectangular microwave cavity
is considered (see Fig.~\ref{fig:8}). Convective boundary conditions
for a constant air temperature of $\nmbr{30}\celcius$ and a convective
heat exchange parameter of $\nmbr{10}\watt/(\m^2\kelvin)$ are applied
on all faces of the block. The incoming mode, ${\bf TE}_{01}$ is exited
at $\nmbr{2.45}\GHz$, providing an input power of $\nmbr{600}\watt$.
Experimental data on the temperature dependent thermal and electromagnetic
material properties of jelly and PTFE are available in \cite{Ma}.
\befig[htb]
\centering
\mbox{\epsfxsize=85mm\epsfbox{oven.eps}} \\[-5pt]
\caption{Geometry of the jelly block and microwave cavity. {\bf (a)}
Vertical cut. {\bf (b)} Horizontal cut.}
\label{fig:8}
\enfig

The application of the method implies several steps. The discretization
according to the {\bf FI} technique is realized with a total of ca.\
$\nmbr{300.000}$ grid points. Then, the higher modes exited in the cavity
are calculated in the time domain. Because of the energy loss in the
jelly block, the steady state is established after $\approx\nmbr{10}\nano$.
Finally, the coupled thermal and electromagnetic calculation is performed
according to the algorithm in Fig.~\ref{fig:3}.

The calculated temperature distribution inside the jelly block
after $180\secs$ of microwave heating is shown in Fig.~\ref{fig:9}b.
Fig.~\ref{fig:9}a shows the temperature distribution taken by
a thermal camera in an experimental setup. Both figures show good
agreement, especially concerning the location of hot spots.
Discrepancies arise mainly because of the inevitable material
cooling before the measurement takes place.
\befig[htb]
\centering
\mbox{\epsfxsize=85mm\epsfbox{t-oven.eps}} \\[-5pt]
\caption{Measurement and simulation of the temperature distribution
inside the jelly block (horizontal cut).}
\label{fig:9}
\enfig

The temperature as a function of time at a fixed point inside
the block is shown in Fig.~\ref{fig:10}. In the same Figure, the
temperature at this point for an uncoupled calculation is also shown.
In the latter, the jelly material retains its thermal and electromagnetic
properties at initial temperature. The final temperature in the uncoupled
model is almost $\nmbr{20}\%$ higher than in the coupled simulation.
\befig[htb]
\centering
\mbox{\epsfxsize=62mm\epsfbox{uncoupled.eps}} \\[-5pt]
\caption{Temperature profile for coupled und uncoupled calculations.}
\label{fig:10}
\enfig
\section{Conclusion}
In the present study a consistent formulation for the calculation
of coupled transient thermal and electromagnetic fields using the
{\bf FI} Method is presented. Due to the weak coupling between both
fields, an iterative algorithm was implemented, which considerably
reduces the calculation effort. The validity of the model was demonstrated
in two microwave heating examples, showing good agreement to analytical
and experimental data, respectively.

\bibliographystyle{IEEE}

\begin{thebibliography}{9}

\bibitem{Hameyer}
J. Driesen, R. Belmans, K. Hameyer, ``Adaptive relaxation algorithms
for thermo-electromagnetic FEM problems'', IEEE Trans. on Magnetics,
Vol.\ 35, No.\ 3, 1999, pp.~1622-1625.

\bibitem{Kost}
L. J\"anicke, A. Kost, ``Convergence properties of of the Newton-Raphson
Method for nonlinear problems'', in Proc.  $\rm XI^{th}$ IEEE COMPUMAG,
Rio de Janeiro, Brazil, PF3-6, 1997, pp.~585-586.

\bibitem{Weiland96e:01}
T. Weiland, ``Time domain electromagnetic field computation with Finite
Difference Methods'', Int.\ J.\ Num.\ Mod., Vol.~9, 1996, pp.~259-319.

\bibitem{RienenPinderWeiland96e:01}
U.~van Rienen, P. Pinder, and T. Weiland, ``Consistent finite integration
approach for coupled calculations of electromagnetic fields and stationary
temperature distributions'', in  Proc.\ $\rm VII^{th}$ IEEE CEFC, Okayama,
1996, pp.~294.

\bibitem{Thoma:Weiland}
P. Thoma, T. Weiland, ``Numerical stability of Finite Difference Time Domain
Methods'', IEEE Trans. on Magnetics, Vol.\ 34, No.\ 5, 1998, pp.~2740-2743.

\bibitem{Pinder}
P. Pinder, ``Zur numerischen Berechnung gekoppelter elektromagnetischer
und thermischer Felder'', PhD Thesis, Darmstadt, 1998.

\bibitem{Jackson:01}
H. Jackson, H. Barmatz, P. Wagner, ``Transient temperature distribution
in a cylinder heated by microwaves'', Mat.\ Res.\ Soc., Spring Meeting,
San Francisco, California, April 8, 1996.

\bibitem{Jackson:02}
H. Jackson H. Barmatz, P. Wagner, priv. comm., 1998.

\bibitem{Fukushima}
L. Fukushima, T. Yamanaka and M. Matsui, J.\ of Japan Soc.\ of Prec.\
Eng.\ {\bf 53}, 1987, pp.~743-748.

\bibitem{Ma} L. Ma, D. Paul, N. Pothecary, C. Railton, ``Experimental
validation of a combined electromagnetic and thermal FDTD model of a
microwave heating process'', IEEE Trans.\ on Microwave Theory and
Techniques, Vol.\ 43, No.\ 11, 1995, pp.~2565-2572.

\end{thebibliography}

\end{document}
